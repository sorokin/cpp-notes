\section{Препроцессирование}

Препроцессирование получает на вход .cpp файл и выдает на выход .i файл. Препроцессирование выполняется программой называемой препроцессор.

Препроцессор проходит по входному файлу, копируя текст в выходной файл. Если он встречает строку начинающуются с символа {\bf \#}, он выполняет команду, которая записана после {\bf \#}. Такие команды называются директивами препроцессора.

Примером директивы препроцессора является директива {\bf \#include}.
\begin{minted}[linenos, frame=lines, framesep=2mm, tabsize = 4, breaklines]{c++}
#include "somefile.h"
\end{minted}

Когда препроцесор встречает эту директиву, он подставляет на её место текст файла {\bf somefile.h}. При этом текст файла {\bf somefile.h} тоже препроцессируется. Если в нём есть какие-то директивы то они будут выполнены.

Пусть даны следующие файлы
\begin{minted}[linenos, frame=lines, framesep=2mm, tabsize = 4, breaklines]{c++}
// print.h
void print(char const*);

// main.cpp
#include "print.h"

int main()
{
    print("Hello, world");
}
\end{minted}
Результатом препроцессирования {\bf main.cpp} будет:
\begin{minted}[linenos, frame=lines, framesep=2mm, tabsize = 4, breaklines]{c++}
void print(char const*);

int main()
{
    print("Hello, world");
}
\end{minted}
В этом можно убедиться выполним команду {\bf g++ -E main.cpp}. Ключ {\bf -E} говорит драйверу, что нужно лишь отпрепроцессировать входной файл.

Директива {\bf \#include} имеет две формы
\begin{minted}[linenos, frame=lines, framesep=2mm, tabsize = 4, breaklines]{c++}
#include <somefile.h>
#include "somefile.h"
\end{minted}

Первая форма ищет файл {\bf somefile.h} в списке каталогов, называемых include directories. Их можно указать компилятору опцией {\bf -I}. Например:
\begin{minted}[linenos, frame=lines, framesep=2mm, tabsize = 4, breaklines]{sh}
g++ -Isome/directory -Ianother/directory -I../and/another/one myfile.cpp
\end{minted}
Файлы ищутся в этих директориях слева направо. Препроцессор автоматически добавляет в конец этого списка пути к директориям в которых лежат хедера стандартной библиотеки.

Вторая форма директивы {\bf \#include} ищет {\bf somefile.h} сначала в текущем каталоге, а потом в include directories.

В стандарте определены следующие директивы препроцессора:
\begin{enumerate}
\item \#define
\item \#elif
\item \#else
\item \#error
\item \#endif
\item \#if
\item \#ifdef
\item \#ifndef
\item \#include
\item \#pragma
\item \#undef
\end{enumerate}
