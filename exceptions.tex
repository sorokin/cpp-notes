\section{Исключения}
\subsection{Введение}
Часто нехватка динамической памяти, неправильный ввод пользователя, ошибка с файловой системой, приводят к тому, что продолжение исполнения логики программы невозможно. Например, если наша функция foo вызывает malloc и malloc вернул ошибку, то функция foo должна завершиться и тоже вернуть ошибку. Возможно, что функция, вызывающая функцию foo, тоже проверит возвращаемое значение и завершится с ошибкой.

В C такая поведение реализовывалось, явной проверкой возвращаемого значения функции с помощью if и исполнением return'а в случае ошибки. C++ имеет встроенный в язык механизм поддержки такого поведения. Этот механизм называется механизмом исключений ({\it exceptions}).

Рассмотрим следующую функцию деления:
\begin{minted}[linenos, frame=lines, framesep=2mm, tabsize = 4, breaklines]{c++}
void div(int a, int b)
{
    return a / b;
}
\end{minted}

Если в эту функцию передать в качестве $b$ $0$, то произойдет undefined behavior. Предположим, что мы хотим, чтобы функция сообщала об ошибке, когда $b = 0$. Для этого сначала необходимо объявить класс исключения, объекты которого будут хранить в себе информацию об исключении.

\begin{minted}[linenos, frame=lines, framesep=2mm, tabsize = 4, breaklines]{c++}
struct division_by_zero
{
    division_by_zero(int dividend, string const &message)
        : dividend(dividend), message(message)
    { }

private:
    int dividend;
    string message;
};
\end{minted}

Теперь можно переписать функцию $div$ следующим образом:

\begin{minted}[linenos, frame=lines, framesep=2mm, tabsize = 4, breaklines]{c++}
void div(int a, int b)
{
    if (b == 0) // возникает исключительное состояние
        throw division_by_zero(a, "in function div(int, int)"); // генерируем исключение.
    return a / b;
}
\end{minted}

Оператор throw завершает исполнение текущей функции и возвращает ошибку в вызывающую функцию. Вызывающая сторона может обработать исключение следующим образом:

\begin{minted}[linenos, frame=lines, framesep=2mm, tabsize = 4, breaklines]{c++}
int main()
{
    int n;
    cin >> n;
    try
    {
        // внутри try-блока указаны операторы, исключения в которых
        // необходимо ловить
        for (int i = 0, a, b; i < n; ++i)
        {
            cin >> a >> b;
            cout << div(a, b);
        }
    }
    catch(division_by_zero const& obj) // здесь указывается тип исключения,
                                       // которое необходимо обработать
    {
        // код обработки исключения
        cout << obj.dividend << "div by 0 " << obj.message();
    }
}
\end{minted}

В данном примере, при завершении div с исключением, цикл for прерывается и исполняется catch-блок, который выводит сообщение об ошибке. После чего функция main завершается.

Если исключения не возникает, то цикл for отработает до конца, catch-блок вызван не будет.

\subsection{Описание конструкций}
Рассмотрим используемые конструкции подробнее. Блок \mintinline{c++}{try-catch} используется для обработки исключений и имеет общий вид:
\begin{minted}[linenos, frame=lines, framesep=2mm, tabsize = 4, breaklines]{c++}
try { /*операторы защищенного блока*/ }
// catch-блоки
catch(exception1 const& e) {/*код обработки*/}
catch(exception2 const& e) {/*код обработки*/}
...
catch(exceptionN const& e) {/*код обработки*/}
\end{minted}
Внутри блока try пишется код, исключения в котором необходимо ловить и обрабатывать. Если изнутри блока try вылетит исключение то, компилятор попытается подобрать подходящий catch-блок. В catch блоке указывается какие действия необходимо сделать, при возникновении исключения указанного типа.

Catch-блок может иметь две формы:
\begin{itemize}
    \item
    \mintinline{c++}{catch(/*declaration*/) { /*обработчик исключения*/ }} ловит исключение указанного или производных от него типов. Переменной исключения можно дать имя. Это позволяет обращаться к пойманному объекту исключения.
    \item
    \mintinline{c++}{catch(...) { /*обработчик исключения*/ }}
    ловит исключения всех типов. В этом случае обращаться к объекту исключения невозможно.
\end{itemize}

Оператор \mintinline{c++}{throw} генерирует исключение. (Иногда говорят, "бросает"\ или "выбрасывает"\ исключение). При генерации исключения происходит следующее:
\begin{enumerate}
    \item
    Создается копия объекта переданного в оператор throw. Эта копия будет существовать до тех пор, пока исключение не будет обработано. Если тип объекта имеет конструктор копирования, то для создания копии будет использован конструктор копирования.
    \item
    Прерывается исполнение программы.
    \item
    Выполняется раскрутка стека, пока исключение не будет обработано.
\end{enumerate}

При раскрутке стека, вызываются деструкторы локальных переменных в обратном порядке их объявления. После разрушения всех локальных объектов текущей функции процесс продолжается в вызывающей функции. Раскрутка стека продолжается пока не будет найден try-catch-блок. При нахождении try-catch-блока, проверяется, может ли исключение быть обработано одним их catch-блоков.

\subsection{Как ловится исключение?}

Catch-блоки проверяются в том порядке, в котором написаны. Обработчик считается подходящим если:
\begin{enumerate}
    \item
    Тип, указанный в catch-блоке, совпадает с типом исключения или является ссылкой на этот тип.
    \item
    Класс, заданный в catch-блоке, является предком класса, заданного в throw, и наследование открытое (public).
    \item
    Указатель, заданный в операторе throw, может быть преобразован по стандартным правилам к указателю, заданному в catch-блоке.
    \item
    В catch-блоке указанно многоточие.
\end{enumerate}

Если найдет нужный catch-блок, то выполняется его код, остальные catch-блоки игнорируются, а выполнение продолжается после try...catch-блока и исключение считается обработанным. Если ни один catch-блок не подошел, процесс раскрутки стека продолжается.

\textcolor{red}{NB}) Так как поиск ведется последовательно, то нужно учитывать порядок catch-блоков (Например, catch(...) должен быть последним).

\textcolor{red}{NB}) Также при наследовании классов исключений следует различать catch(type\& obj) и catch(type obj). В первом случае obj ссылается на этот объект и копии не создается. Во втором случае при входе в catch блок делается копия объекта-исключения, вследствие чего мы теряем возможность вызывать виртуальные функции.

Пример:
\begin{minted}[linenos, frame=lines, framesep=2mm, tabsize = 4, breaklines]{c++}
struct base
{
    virtual char const* msg() const
    {
        return "base";
    }
};

struct derived : base
{
    virtual char const* msg() const
    {
        return "derived";
    }
};

void wrong()
{
    try
    {
        throw derived();
    }
    catch (base e)
    {
        std::cout << e.msg() << std::endl;
    }
}

void right()
{
    try
    {
        throw Exception_derived();
    }
    catch (base const& e)
    {
        std::cout << e.msg() << std::endl;
    }
}
\end{minted}

В данном примере функция right() выводит <<derived>>, а функция wrong() выводит <<base>>, поскольку внутри catch объект исключения был скопирован и новая копия имеет тип base.

В некоторых случаях внутри catch-блока может быть необходимо не завершать раскрутку стека. Для этого существует специальная форма оператора throw без аргумента. Она означает проброс текущего исключения с сохранением его динамического типа.

\begin{minted}[linenos, frame=lines, framesep=2mm, tabsize = 4, breaklines]{c++}

struct base
{
    virtual char const* msg() const
    {
        return "base";
    }
};

struct derived : base
{
    virtual char const* msg() const
    {
        return "derived";
    }
};

void third()
{
    throw derived();
}

void second()
{
    try
    {
        third();
    }
    catch (base const &obj)
    {
        std::cout << obj.msg() << std::endl;
        throw; //пробрасываем
        // здесь obj имеет тип Exception_base.
        // но пробрасывается дальше исходное исключение имеющее тип Exception_derived.
    }
}

void first()
{
    try
    {
        second();
    }
    catch (derived const &obj)
    {
        std::cout << obj.msg() << std::endl;
    }
}

int main()
{
    first();
    return 0;
}


\end{minted}

\textbf{Вывод программы:} \\
> derived \\
> derived \\

Статический тип объекта-параметра в текущем catch-блоке может отличаться от динамического типа исключения, но throw без аргументов пробрасывает текущее исключение сохраняя его динамический тип.

\textcolor{red}{NB})Если необходимо изменить тип исключения, то внутри catch-блока возможно кинуть исключение нового типа.

\begin{minted}[linenos, frame=lines, framesep=2mm, tabsize = 4, breaklines]{c++}
void load_config()
{
    try
    {
        // ...
    }
    catch (file_not_found const&)
    {
        throw config_loading_failure();
    }
}
\end{minted}

\subsection{Function-try-block}

Предположим есть класс, в конструкторе которого мы хотим ловить и обрабатывать исключения.
\begin{minted}[linenos, frame=lines, framesep=2mm, tabsize = 4, breaklines]{c++}
struct mytype
{
public:
    mytype()
        : member()
    {
        try
        {
            // Constructor's code
        }
        catch (...)
        {
            // ...
        }
    }
private:
    member_type member;
}
\end{minted}

Заметим, что вызов конструкторов членов не находится внутри try-блока и исключения возникшие в их конструкторах не поймаются. Для ловли исключений из конструкторов членов существует специальный синтаксис называющийся {\it function-try-block}:

\begin{minted}[linenos, frame=lines, framesep=2mm, tabsize = 4, breaklines]{c++}
struct mytype
{
    mytype()
    try : member()
    {
        // Constructor's code
    }
    catch (...)
    {

    } // implicit throw
private:
    member_type member;
}
\end{minted}

У функциональных try-блоков в конструкторах, есть особенность: они всегда бросают исключение повторно. Это связано с тем, что при ошибке создания члена, весь объемлющий объект оказывается не до конца созданным и продолжать работу невозможно.

Function-try-block может использоваться и с обычными функциями, в этом случае он эквивалентен оборащиванию тела в try...catch блок то есть он не пробрасывает исключение наверх автоматически.

\begin{minted}[linenos, frame=lines, framesep=2mm, tabsize = 4, breaklines]{c++}
int main()
try {
    // main's body
}
catch (...)
{}
\end{minted}

\subsection{Уничтожение объекта при исключении в конструкторе}

При исполнении конструктора класса, вызываются конструкторы всех членов этого класса. Если исключение возникает при создании одного из членов класса, то в процессе раскрутки стека будут вызваны деструкторы от всех уже созданных членов. У самого объекта деструктор не вызывается, так как объект не считается созданным пока его конструктор не отработал полностью. Если конструктор захватывает некоторые ресурсы и потом бросается исключение, то конструктору следует самостоятельно освободить эти ресурсы перед выбрасыванием исключения.

\subsection{Исключения в деструкторах}

Как правило деструктор вызывается, чтобы произвести освобождение ресурсов и вызывающая сторона не заинтереснована в получении исключения. Начиная с C++11 деструкторы по умолчанию помечаются как \mintinline{c++}{noexcept}.

Теоретически из деструктора можно бросать исключения при этом нужно пометить его явно как \mintinline{c++}{noexcept(false)}. Иначе проброска исключения из \mintinline{c++}{noexcept} функции проведет в вызову \mintinline{c++}{std::terminate()} и завершению программы.

Следует так же иметь ввиду, что если деструктор был вызван не при штатном исполнении программы, а при раскрутке стека, то исключение вылетевшее из такого деструктора приведет к вызову \mintinline{c++}{std::terminate()}.

\subsection{\mintinline{c++}{std::terminate()}}

\mintinline{c++}{std::terminate()} --- это функция, которая вызывается, чтобы завершить программу, в случаях нерьезных ошибок использования исключений. Она вызывается:
\begin{itemize}
\item Если исключение брошено и не поймано ни одним catch-блоком, то есть пробрасывается наружу из main().
\item Если исключение пробрасывается наружу из созданного потока.
\item Если во время обработки исключения десктруктор, вызванный при раскрутке стека, бросает исключение и оно вылетает наружу.
\item Если функция переданная в \mintinline{c++}{std::atexit} и \mintinline{c++}{std::at_quick_exit} бросает исключение.
\item Если функция нарушает гарантии noexcept specification. Например, если функция помеченная как noexcept бросает исключение.
\item Если конструктор или деструктор статического или локального для потока объекта бросает исключение.
\end{itemize}

По умолчанию \mintinline{c++}{std::terminate()} вызывает \mintinline{c++}{std::abort()}, но можно это изменить, написав свою функцию и зарегистрировав ее с помощью функции \mintinline{c++}{std::set_terminate()}.
