\section{Exceptions}
\subsection{Введение}
Часто нехватка динамической памяти, неправильный ввод пользователя, ошибка с файловой системой, приводят к тому, что продолжение исполнения логики программы невозможно. Например, если наша функция foo вызывает malloc и malloc вернул ошибку, то функция foo должна завершиться и тоже вернуть ошибку. Возможно, что функция, вызывающая функцию foo, тоже проверит возвращаемое значение и завершится с ошибкой.

В C такая поведение реализовывалось, явной проверкой возвращаемого значения функции с помощью if и исполнением return'а в случае ошибки. C++ имеет встроенный в язык механизм поддержки такого поведения. Этот механизм называется механизмом исключений.

Рассмотрим следующую функцию деления:
\begin{minted}[linenos, frame=lines, framesep=2mm, tabsize = 4, breaklines]{c++}
void div(int a, int b) {
    return a / b;
}
\end{minted}

Если в эту функцию передать в качестве $b$ $0$, то произойдет undefined behavior. Предположим, что мы хотим, чтобы функция сообщала об ошибке, когда $b = 0$. Для этого сначала необходимо объявить класс исключения, объекты которого будут хранить в себе информацию об исключении.

\begin{minted}[linenos, frame=lines, framesep=2mm, tabsize = 4, breaklines]{c++}
class Division_by_zero {
    int dividend
    string message;
    Division_by_zero(int dividend, string const &message) :
        dividend(dividend), message(message) { }
};
\end{minted}

Теперь можно переписать функцию $div$ следующим образом:

\begin{minted}[linenos, frame=lines, framesep=2mm, tabsize = 4, breaklines]{c++}
void div(int a, int b) {
    if (b == 0) // возникает исключительное состояние
        throw Division_by_zero(a, "in function div(int, int)"); // генерируем исключение.
    return a / b;
}
\end{minted}

Оператор throw завершает исполнение текущей функции и возвращает ошибку в вызывающую функцию. Вызывающая сторона может обработать исключение следующим образом:

\begin{minted}[linenos, frame=lines, framesep=2mm, tabsize = 4, breaklines]{c++}
int main() {
    int n;
    cin >> n;
    try { // здесь указываем операторы, в которых мы хотим ловить исключения.
        for (int i = 0, a, b; i < n; ++i) {
            cin >> a >> b;
            cout << div(a, b);
        }
    } catch(Division_by_zero const& obj) { //здесь указываем тип исключения, которое мы хотим обработать
    // здесь обрабатываем исключение
        cout << obj.dividend << "div by 0 " << obj.message();
    }
}
\end{minted}

В данном примере, при завершении div с исключением, цикл for прерывается и исполняется catch-блок, который выводит сообщение об ошибке. После чего функция main завершается.

Если исключения не возникает, то цикл for отработает до конца, catch-блок вызван не будет.

\subsection{Описание конструкций}
Рассмотрим используемые конструкции подробнее. Блок \mintinline{c++}{try-catch} используется для обработки исключений и имеет общий вид:
\begin{minted}[linenos, frame=lines, framesep=2mm, tabsize = 4, breaklines]{c++}
try { /*операторы защищенного блока*/ }
// catch-блоки
catch(exception1 const& e) {/*код обработки*/}
catch(exception2 const& e) {/*код обработки*/}
...
catch(exceptionN const& e) {/*код обработки*/}

\end{minted}
Внутри блока try пишется код, исключения в котором необходимо ловить и обрабатывать. Если изнутри блока try вылетит исключение то, компилятор попытается подобрать подходящий catch-блок. В catch блоке указывается какие действия необходимо сделать, при возникновении исключения указанного типа.

Catch-блок может иметь две формы:
\begin{itemize}
    \item
    \mintinline{c++}{catch(/*declaration*/) { /*обработчик исключения*/ }} Ловит исключение указанного или производных от него типов. Переменной исключения можно дать имя. Это позволяет обращаться к пойманному объекту исключения.
    \item
    \mintinline{c++}{catch(...) { /*обработчик исключения*/ }}
    Ловит исключения всех типов. В этом случае обращаться к объекту исключения невозможно.
\end{itemize}

Оператор \mintinline{c++}{throw} генерирует исключение. (Иногда говорят, "бросает"\ или "выбрасывает"\ исключение). При генерации исключения происходит следующее:
\begin{enumerate}
    \item
    Создается копия объекта переданного в оператор throw. Эта копия будет существовать до тех пор, пока исключение не будет обработано. Если тип объекта имеет конструктор копирования, то для создания копии будет использован конструктор копирования.
    \item
    Прерывается исполнение программы.
    \item
    Выполняется раскрутка стека, пока исключение не будет обработано.
\end{enumerate}

При раскрутке стека, вызываются деструкторы локальных переменных в обратном порядке их объявления. После разрушения всех локальных объектов текущей функции процесс продолжается в вызывающей функции. Раскрутка стека продолжается пока не будет найден try-catch-блок. При нахождении try-catch-блока, проверяется, может ли исключение быть обработано одним их catch-блоков.

\subsection{Как ловится исключение?}

Catch-блоки проверяются в том порядке, в котором написаны. Обработчик считается подходящим если:
\begin{enumerate}
    \item
    Тип, указанный в catch-блоке, совпадает с типом исключения или является ссылкой на этот тип.
    \item
    Класс, заданный в catch-блоке, является предком класса, заданного в throw, и наследование открытое (public).
    \item
    Указатель, заданный в операторе throw, может быть преобразован по стандартным правилам к указателю, заданному в catch-блоке.
    \item
    В catch-блоке указанно многоточие.
\end{enumerate}

Если найдет нужный catch-блок, то выполняется его код, остальные catch-блоки игнорируются, а выполнение продолжается после try...catch-блока и исключение считается обработанным. Если ни один catch-блок не подошел, процесс раскрутки стека продолжается.

\textcolor{red}{NB}) Так как поиск ведется последовательно, то нужно учитывать порядок catch-блоков (Например, catch(...) должен быть последним).

\textcolor{red}{NB}) Также при наследовании классов исключений следует различать catch(type\& obj) и catch(type obj). В первом случае obj ссылается на этот объект и копии не создается. Во втором случае при входе в catch блок делается копия объекта-исключения, вследствие чего мы теряем возможность вызывать виртуальные функции.

Пример:
\begin{minted}[linenos, frame=lines, framesep=2mm, tabsize = 4, breaklines]{c++}
struct Exception_base {
    virtual char const* msg() const {
        return "base";
    }
};

struct Exception_derived : Exception_base {
    virtual char const* msg() const {
        return "derived";
    }
};

int f() {
    try {
        throw Exception_derived();
    }
    catch (base e) {
        std::cout << e.msg() << std::endl;
    }
}

int g() {
    try {
        throw Exception_derived();
    }
    catch (base const& e) {
        std::cout << e.msg() << std::endl;
    }
}
\end{minted}

В данном примере g() выводит <<derived>>, а функция f() выводит <<base>>, поскольку объект исключения был скопирован с базы объекта, который мы передали в оператор throw и новая копия имеет тип base.

В некоторых случаях внутри catch-блока может быть необходимо не завершать раскрутку стека. Для этого существует специальная форма оператора throw без аргумента. Она означает проброс текущего исключения с сохранением его типа.

\begin{minted}[linenos, frame=lines, framesep=2mm, tabsize = 4, breaklines]{c++}

struct Exception_base {
    virtual char const* msg() const {
        return "base";
    }
};

struct Exception_derived : Exception_base {
    virtual char const* msg() const {
        return "derived";
    }
};

void third() {
    throw Exception_derived();
}

void second() {
    try {
        third();
    }
    catch (Exception_base const &obj) {
        std::cout << obj.msg() << std::endl;
        throw; //пробрасываем
        // здесь obj имеет тип Exception_base.
        // но это не мешает поймать потом тоже исключение как Exception_derived.
    }
}

void first () {
    try {
        second();
    }
    catch (Exception_derived const &obj) {
        std::cout << obj.msg() << std::endl;
    }
}

int main() {
    first();
    return 0;
}


\end{minted}

\textbf{Вывод программы:} \\
> derived \\
> derived \\

Значит тип объекта-параметра в текущем catch-блоке может отличаться от типа исключения, и это не влияет на дальнейшую обработку исключения в других catch-блоках.

 Например, это позволяет найти выход из такой ситуации: мы захотели вставить куда-то в глубь уже написанного кода try-catch-блок для логирования всех исключений. Тогда будем ловить по типу Exception\_all, который сделаем предком всех наших исключений. Ловим и пробрасывать дальше, чтобы не нарушать обработку производный от него исключений.

 \textcolor{red}{NB})Если необходимо изменить тип исключения, то мы можем в конце catch-блока сразу кинуть новое исключение нового типа.

\begin{minted}[linenos, frame=lines, framesep=2mm, tabsize = 4, breaklines]{c++}
int main()
{
    try {
        try {
            throw derived();
        }
        catch (base const& e) {
            throw e;
        }
    }
    catch (base const& e) { ... }
}
\end{minted}

\subsection{Function-try-block}

Часто мы хотим, чтобы все тело функции находилось в try-блоке. Тогда это try-блок называется функциональным. И для него есть отдельный синтаксис.
\begin{minted}[linenos, frame=lines, framesep=2mm, tabsize = 4, breaklines]{c++}
int main() try {
    //main's body
}
catch (...) { }
\end{minted}

Здесь функциональные try-блоки являются синтаксическим сахаром, но есть ситуации когда без них не обойтись: обработка исключений в конструкторе.

Вот есть класс, котором мы хотим ловить и обрабатывать исключения.
\begin{minted}[linenos, frame=lines, framesep=2mm, tabsize = 4, breaklines]{c++}
class St {
public:
    St(): member() {
        try {
            // Constructor's code
        }
        catch (...) { }
    }
private:
    Member_type member;
}
\end{minted}

Но заметим, что вызов конструкторов членов не находится внутри try-блока и исключения возникшие в их конструкторах не поймаются.
Поэтому мы используем здесь функциональный try-блок:

\begin{minted}[linenos, frame=lines, framesep=2mm, tabsize = 4, breaklines]{c++}
class St {
public:
    St() try: member() {
        // Constructor's code
    }
    catch (...) {

    } // implicit throw
}
private:
    Member_type member;
}
\end{minted}

У функциональных try-блоков в конструкторах, есть особенность: они всегда бросают исключение повторно.

\subsection{Уничтожение объекта при исключении в конструкторе}

Если возникновении исключения в конструкторе некоторого объекта, объект не считается созданным и деструктор для него не вызывается. Если конструктор захватывает некоторые ресурсы и потом бросается исключение, то конструктору следует самостоятельно освободить эти ресурсы перед выбрасыванием исключения.

При исполнении конструктора класса, вызываются конструкторы всех членов этого класса. Если исключение возникает при создании одного из членов класса, то в процессе раскрутки стека будут вызваны деструкторы от всех уже созданных членов. У самого объекта деструктор не вызывается, так как объект не считается созданным пока его конструктор не отработал полностью.

\subsection{Best practice}
\begin{itemize}
\item
Для корректной работы с ресурсами необходимо учитывать существование исключений. То есть, если возникает исключение и некоторые из ресурсов еще не были освобождены, то следует поймать исключение и освободить оставшиеся ресурсы.

Например, мы пишем конструктор копирования для вектора, и нам необходимо скопировать данные в другой участок памяти. При этом если во время копирование какого-то объекта возникнет исключение, то обязательно уже созданные объекты должны быть разрушены. Причем иногда важно делать это в обратном порядке их создания.
\begin{minted}[linenos, frame=lines, framesep=2mm, tabsize = 4, breaklines]{c++}
template <typename T>
void copy_construct(T* dist, T const * sourse, size_t size) {
    size_t i = 0;
    try {
        for (; i != size; ++i) {
            new (dist + i) T(sourse[i]);
        }
    }
    catch (...) { // если ошибка при копировании
        for (size_t j = i; j != 0; --j) {
            dist[j - 1].~T(); // вызовем деструкторы созданных объектов
        }
        throw;
    }
}
\end{minted}

\item
Полезно знать про стандартные исключения, такие как  \mintinline{c++}{std::bad_alloc}, \mintinline{c++}{std::bad_cast}, \mintinline{c++}{std::bad_typeid} и т. д. Они связанны иерархией наследования и имеют общего предка \mintinline{c++}{std::exception}.

Подробнее можно почитать здесь: \\
\url{https://www.tutorialspoint.com/cplusplus/cpp_exceptions_handling.htm} \\
\url{http://en.cppreference.com/w/cpp/error/exception} \\

\item
Хорошим тоном является наследование от \mintinline{c++}{std::exception}.

\item
Когда мы организовываем исключения в иерархии классов, то получаем мощный механизм описания исключение и способов их обработки. Создав такую структуры мы можем обрабатывать как более общие ошибки, так и более специализированные.

Пример:
\begin{minted}[linenos, frame=lines, framesep=2mm, tabsize = 4, breaklines]{c++}
    class StackException {};
    class popOnEmpty(): public StackException {};
    class pushOnFull(): public StackException {};
\end{minted}
	Причем сгенерировав исключение типа \mintinline{c++}{popOnEmpty}, мы можем в разных обработчиках независимо выбирать: обработать как \mintinline{c++}{popOnEmpty} или как \mintinline{c++}{StackException}, так как тип исключения не теряется при повторной генерации этого исключения.

\item Виртуальное наследование классов исключений:

Рассмотрим иерархию исключений для кода, которому необходимо обрабатывать ошибки ввода.
\begin{minted}[linenos, frame=lines, framesep=2mm, tabsize = 4, breaklines]{c++}
    class InterfaceException {};
    class InputException(): public virtual InterfaceException {};
    class OutputException(): public virtual InterfaceException {};
    class InputOrOutputException: public InputException, public OutputException {};
\end{minted}
    Важно что мы наследуемся от \mintinline{c++}{InterfaceException} виртуально, так как иначе \mintinline{c++}{catch(InterfaceException const&)} не будет ловить \mintinline{c++}{InputOrOutputException}. Это связано с тем, что путь наследования без виртуального наследования будет не опреден.

\item
Если необходимо бросать и ловить исключения из деструктора, то нужно помнить, что начиная с C++11, все деструкторы неявно помечены как \mintinline{c++}{noexcept}. Из-за этого необходимо явно указать, что дектруктор может бросать. Если этого не сделать, то при попытке выбросить исключение из деструктора вызовется функция \mintinline{c++}{std::terminate()}.

Пример:
\begin{minted}[linenos, frame=lines, framesep=2mm, tabsize = 4, breaklines]{c++}
    class A {
        ~A() noexcept(false) {
            //...
        }
    };
\end{minted}

\end{itemize}
\subsection{Bad practice}
\begin{itemize}
\item
Код использующий исключения такой же быстрый как и код без них, если исключения не возникают. Если же исключения возникают, то в современных реализациях они обычно дороже проверок на коды возврата, поэтому не рекомендуется использовать исключения как часть обычного control-flow.
\end{itemize}

\subsection{std::terminate()}

Это функция, которая вызывается если у механизма исключений не получается корректно отработать, чтобы завершить программу.
Случаи когда она вызывается:
\begin{itemize}
\item Если исключение брошено и не поймано ни одним catch-блоком, то есть пробрасывается вне main().
\item Если во время обработки исключения десктруктор, вызванный при раскрутке стека, бросает исключение и оно вылетает наружу.
\item Если функция переданная в \mintinline{c++}{std::atexit} и \mintinline{c++}{std::at_quick_exit} бросит исключение.
\item Если функция нарушит гарантии noexcept specification. Например, если функция помеченная как noexcept бросит исключение.
\item Если конструктор или деструктор статического или локального для одного треда объект бросает исключение.
\end{itemize}

По умолчанию \mintinline{c++}{std::terminate()} просто вызывает \mintinline{c++}{std::abort()}, но можно это изменить, написав свою функцию \mintinline{c++}{my_terminate()} и зарегистриров ее как терминальную.
\begin{minted}[linenos, frame=lines, framesep=2mm, tabsize = 4, breaklines]{c++}
void my_terminate() {
    cout << "It's not a bug, it's a feature!";
    std::abort();
}
/**/
set_terminate(my_teminate);
\end{minted}
