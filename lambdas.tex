\section{Анонимные функции}
\subsection{Передача функции в качестве параметра}
Иногда требуется передать функцию в качестве параметра в другую функцию.
Пример такой ситуации~--- вызов \mintinline{c++}{std::sort} с нестандартным компаратором.

\mintinline{c++}{std::sort} принимает третьим параметром объект, у которого определён \mintinline{c++}{operator ()}.
Тип этого объекта задаётся параметром шаблона. \mintinline{c++}{std::sort} вызывает этот оператор у объекта, чтобы произвести сравнение.

В качестве такого объекта может выступать указатель на функцию:
\begin{minted} [linenos, frame=lines, framesep=2mm, tabsize = 4, breaklines]{c++}
    bool cmp(int x, int y) {
        return x % 10 < y % 10;
    }
    std::vector<int> a;
    // ... 
    std::sort(a.begin(), a.end(), cmp);
\end{minted}

Сама функция \mintinline{c++}{std::sort} выглядит следующим образом:
\begin{minted} [linenos, frame=lines, framesep=2mm, tabsize = 4, breaklines]{c++}
    template <class Iterator, class Compare>
    void sort (Iterator first, Iterator last, Compare comp) {
        // ...
        if (comp(x, y)) {
            // ...
        }
        // ...
    }
\end{minted}

\subsubsection{Функциональные объекты}
Может потребоваться передать в передаваемую функцию какие-то дополнительные параметры.
Например, нужно отсортировать числа, как в предыдущем примере, но вместо фиксированного модуля $10$
требуется использовать значение некоторой переменной. Чтобы сделать это, передавая указатель на функцию,
придётся сделать эту переменную глобальной:
\begin{minted} [linenos, frame=lines, framesep=2mm, tabsize = 4, breaklines]{c++}
    int k;
    bool cmp(int x, int y) {
        return x % k < y % k;
    }
    // ... 
    k = 10;
    std::sort(a.begin(), a.end(), cmp);
\end{minted}

У такого подхода есть ряд недостатков: создаётся глобальная переменная, которая используется только при вызове
\mintinline{c++}{std::sort}, невозможно выполнять этот код многопоточно с разными значениями $k$.

Вместо этого можно передать объект, который содержит в себе нужную переменную и имеет \mintinline{c++}{operator ()}:
\begin{minted} [linenos, frame=lines, framesep=2mm, tabsize = 4, breaklines]{c++}
    struct Cmp {
        int k;
        Cmp(int nk): k(nk) {}
        bool operator () (int x, int y) const {
            return x % k < y % k;
        }
    };
    int k = 10;
    std::sort(a.begin(), a.end(), Cmp(k));
\end{minted}

Если требуется изменять изнутри компаратора локальные переменные функции, вызывающей \mintinline{c++}{std::sort}, можно
сохранить в этот объект ссылки на них.
\begin{minted} [linenos, frame=lines, framesep=2mm, tabsize = 4, breaklines]{c++}
    struct Cmp {
        int k;
        int &cnt;
        Cmp(int nk, int &ncnt): k(nk), cnt(ncnt) {}
        bool operator () (int x, int y) const {
            ++cnt;
            return x % k < y % k;
        }
    };
    int cnt = 0;
    int k = 10;
    std::sort(a.begin(), a.end(), Cmp(k, cnt));
\end{minted}

Недостаток такого подхода~--- громоздсксть. Приходится писать целый класс, у которого содержательная часть~--- только
\mintinline{c++}{operator ()}. C++11 предоставляет способ создавать функциональные объекты меньшим количеством кода.

\subsection{Синтаксис лямбда-функций}
Лямбда-функции (или анонимные функции)~--- это способ создавать функцинальные объекты в C++11.
Пример из предыдущего раздела может быть реализован следующим образом с применением лямбда-функции:

\begin{minted} [linenos, frame=lines, framesep=2mm, tabsize = 4, breaklines]{c++}
    int cnt = 0;
    int k = 10;
    std::sort(a.begin(), a.end(), [k, &cnt] (int x, int y) {
        ++cnt;
        return x % k < y % k;
    });
\end{minted}

Такой код создаёт анонимный класс, аналогичный классу \mintinline{c++}{Cmp} из примера выше, и передаёт его эксемпляр в
\mintinline{c++}{std::sort}.

Синтаксис лямбда-функции выглядит следующим образом:

\mintinline{c++}{[capture list] (params) specifiers exception -> ret { body }}

\begin{itemize}
    \item \emph{capture list}~--- список переменных, которые будут доступны изнутри лямбды. Переменные могут передаваться
    по значению и по ссылке.
    \begin{itemize}
        \item \mintinline{c++}{[]}~--- квадратные скобки нужно писать, даже если capture list пустой.
        \item \mintinline{c++}{[a,b,c]}~--- захват переменных по значению.
        \item \mintinline{c++}{[&a,&b,&c]}~--- захват переменных по ссылке.
        \item \mintinline{c++}{[a,&b,c]}~--- захват по ссылке и по значению можно смешивать.
        \item \mintinline{c++}{[=]}~--- захват всех переменных, к которым обращается лямбда, по значению.
        \item \mintinline{c++}{[&]}~--- то же самое, но по ссылке.
        \item \mintinline{c++}{[&,a,b]}~--- захват $a$ и $b$ по значению, а всего остального~--- по ссылке.
        \item \mintinline{c++}{[=,&a,&b]}~--- захват $a$ и $b$ по ссылке, а всего остального~--- по значению.
        \item \mintinline{c++}{[this]}~--- захват \mintinline{c++}{this} той функции, в которая объявлена лямбда.
        \mintinline{c++}{[=]} и \mintinline{c++}{[&]} тоже захватывают \mintinline{c++}{this}.
        \item C++14: \mintinline{c++}{[x = 2 + 3]}~--- то же самое, что и захват по значению новой переменной $x$ с
        заданным значением. Можно захватывать move-only типы: \mintinline{c++}{[x = std::move(x)]}
        \item C++14: \mintinline{c++}{[&x = a]}~--- захват ссылки на $a$ с именем $x$. Вместо $a$ может быть не только
        переменная, но и любая ссылка, например, на элемент массива.
    \end{itemize}
    Переменные, захваченные по значению, нельзя изменять изнутри лямбды, если не указан спецификатор \mintinline{c++}{mutable}.
    \item \emph{params}~--- список параметров функции.
    Значения параметров по умолчанию не разрешены до C++14.
    \item \emph{specifiers}:
    \begin{itemize}
        \item \mintinline{c++}{mutable}~--- разрешает изменять переменные, захваченные по значению, и вызывать у них
        не-const методы. Поскольку при захвате по значаению переменные копируются, их изменение не повлияет на значения
        переменных, которые были захвачены.
    \end{itemize}
    \item \emph{exception}~--- можно указать спецификаторы \mintinline{c++}{throw}, \mintinline{c++}{noexcept},
    как у обычных функций.
    \item \emph{ret}~--- тип возвращаемого значения. Если не указан, то тип будет выведен автоматически. Также, если тип
    не указан, а параметров функция не прининмает, можно не писать круглые скобки для параметров.
\end{itemize}

\subsection{Устройство функционального объекта}
Каждая лямбда-функция~--- это отдельный тип, называемый \emph{closure type}. Он содержит

\begin{itemize}
    \item \mintinline{c++}{operator ()} с заданными параметрами и типом возвращаемого значения.
    Если лямбда не имеет спецификатора \mintinline{c++}{mutable}, оператор является \mintinline{c++}{const}.
    \item Конструктор копирования и move-конструктор. Конструктор по умолчанию отсутствует.
    \item \mintinline{c++}{ClosureType& operator=(const ClosureType&) = delete;}
    \item Члены класса: захваченные по значению переменные, ссылки на захваченные по ссылке переменные.
    \item Если capture list пустой, то closure type может неявно приводиться к указателю на функцию с соответствующей сигнатурой.
\end{itemize}

\subsection{Сохранения лямбда-функций в переменные}

Лямбда-функцию можно не только куда-то передать, но и сохранить в переменную или вернуть из функции.
Поскольку тип лямбды не имеет имени, для объявления переменной или функции следует использовать \mintinline{c++}{auto}.

Обратите внимание, что захват переменных по ссылке не продлевает их время жизни. Например, следующий код
будет работать неправильно:
\begin{minted} [linenos, frame=lines, framesep=2mm, tabsize = 4, breaklines]{c++}
    auto foo() {
        std::vector<int> v;
        return [&v] {
            // ...
        };
    }
\end{minted}
Вектор $v$ будет уничтожен при выходе из функции, поэтому обращение к нему из лямбды приведёт к
неопределённому поведению. В этом случае необходимо захватывать по значению.

При копировании лямбды копируются все захваченные по значению переменные.

Поскольку каждая лямбда~--- отдельный тип, возникают определённые трудности:
\begin{itemize}
    \item Нельзя завести переменную, которая может хранить лямбду не какого-то фиксированного типа, или завести массив разных лямбд.
    \item Нельзя вернуть из функции лямбду не фиксированного типа.
    \item Если нужно принять лямбду в функцию, приходится делать её шаблонной.
\end{itemize}
Те же проблемы есть и у обычных функциональных объектов. Их можно разрешить при помощи \mintinline{c++}{std::function}.

\subsubsection{std::function}

\mintinline{c++}{std::function}~--- это класс, принимающий в качестве парамерв шаблона типы аргументов и возвращаемого
значения функции. В него можно записать любой объект с оператором \mintinline{c++}{()} с заданной сигнатурой:
лямбду, обычный функциональный объект или указатель на функцию. Класс похож на \mintinline{c++}{std::any},
но предназначен для хранения функциональных объектов и имеет \mintinline{c++}{operator ()}.

Поскольку функциональные объекты могут иметь разный размер, \mintinline{c++}{std::function} может выделять для них
дополнительную память.

Поскольку \mintinline{c++}{std::function}~--- один тип для всех функцинальных объектов с заданной сигнатурой,
в переменную такого типа можно сохранить любой подходящий функциональный объект, их можно без проблем
складывать в коллекции или возвращать из функций.

\begin{minted} [linenos, frame=lines, framesep=2mm, tabsize = 4, breaklines]{c++}
    #include <functional> // std::function определён в этом заголовочном файле

    int bar(int a, int b) {
        return a * b;
    }

    int main() {
        int n = 5;
        std::function<int(int, int)> foo = [n](int x, int y) {
            return x * n + y;
        };
        std::function<double(double)> foo2 = [n] (double x) { return n * x; };

        foo = bar; // Можно хранить в std::function обычные указатели на функции

        // std::function можно складывать в коллекции
        std::vector<std::function<double(double)>> fs = {cos, sin, foo2};
        std::cout << fs[0](0.1) << "\n";
        std::cout << fs[1](0.1) << "\n";
        std::cout << fs[2](0.1) << "\n";
    }
\end{minted}

\mintinline{c++}{std::function} можно копировать. Неинициализированный \mintinline{c++}{std::function}
не содержит никакого функционального объекта, при попытке вызова бросается \mintinline{c++}{std::bad_function_call}.
Сделать \mintinline{c++}{std::function} пустым можно, присвоив ему \mintinline{c++}{nullptr},
также с \mintinline{c++}{nullptr} можно сравнивать.

\subsubsection{Рекурсивный вызов лямбда-функции}
Следующий код не скомпилируется:
\begin{minted} [linenos, frame=lines, framesep=2mm, tabsize = 4, breaklines]{c++}
    auto fact = [&fact] (int n) -> int {
        if (n == 0)
            return 1;
        return n * fact(n - 1);
    };
    std::cout << fact(5) << "\n";
\end{minted}

Это происходит из-за того, что тип переменной $fact$ ещё не выведен в момент её использования.
Чтобы можно было вызвать лямбду рекурсивно, нужно сохранить её в \mintinline{c++}{std::function}:
\begin{minted} [linenos, frame=lines, framesep=2mm, tabsize = 4, breaklines]{c++}
    std::function<int(int)> fact = [&fact] (int n) -> int {
        if (n == 0)
            return 1;
        return n * fact(n - 1);
    };
    std::cout << fact(5) << "\n";
\end{minted}

Обратите внимание, что $fact$ захватывается по ссылке. В этом случае нельзя захватывать по значению, так
как в момент захвата переменной $fact$ ещё не присвоено значение.

Если нужно объявить несколько лямбда-функций, которые вызывают друг друга, нужно разделить объявление переменной и
призваивание ей значения:
\begin{minted} [linenos, frame=lines, framesep=2mm, tabsize = 4, breaklines]{c++}
    std::function<int(int)> foo2;
    std::function<int(int)> foo1 = [&] (int n) -> int {
        if (n == 0)
            return 1;
        return n * foo2(n - 1);
    };
    foo2 = [&] (int n) -> int {
        if (n == 0)
            return 1;
        return 5 + foo1(n - 1);
    };
    std::cout << foo2(7) << "\n";
\end{minted}

В этом примере $foo1$ можно сделать не \mintinline{c++}{std::function}, а просто \mintinline{c++}{auto},
так как она не используется изнутри себя.

\subsection{Примеры}

\subsubsection{Поиск в глубину}
\begin{minted} [linenos, frame=lines, framesep=2mm, tabsize = 4, breaklines]{c++}
    std::vector<int> dfs(std::vector<std::vector<int>> const& e) {
        std::vector<int> tin(e.size(), -1);
        int t = 0;
        std::function<void(int)> go = [&] (int v) {
            if (tin[v] != -1)
                return;
            tin[v] = t++;
            for (int nv : e[v])
                go(nv);
        };
        go(0);
        return tin;
    }
\end{minted}

\subsubsection{Упрощённый bind}
\begin{minted} [linenos, frame=lines, framesep=2mm, tabsize = 4, breaklines]{c++}
    template < typename Callable, typename... Ps >
    auto bind(Callable f, Ps... ps) {
        return [f, ps...] () {
            return f(ps...);
        };
    }
\end{minted}
Этот пример демонстрирует использование лямбд с variadic templates, а также возвращение лямбд из функций.
