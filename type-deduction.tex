	\section{Вывод типов}
	До \textit{С++11} был один набор правил вывода типов - для шаблонов функций. В \textit{С++11} появились два новых, из-за появления \mintinline{c++}{auto} и \mintinline{c++}{decltype}.
	Все более широкое применение выводов типов освобождает от нужды писать очевидные или излишние типы. Давайте разберем отдельно
	
	\subsection{Вывод типов шаблона}
		В чем прелесть знания данного пункта? Он является базой для одной из самых привлекательных фич \textit{С++11} - \mintinline{c++}{auto}.
		
		Пусть в данном пункте у нас будут такие обозначения.
		
\begin{minted}[linenos, frame=lines, framesep=2mm, tabsize = 4, breaklines]{c++}
template<typename T>
void f(ParamType param)
// и будем вызывать f(expr)
\end{minted}

		\subsubsection{Указатели}
			Давайте начнем с указателей. Пусть у нас есть такая функция:
\begin{minted}[linenos, frame=lines, framesep=2mm, tabsize = 4, breaklines]{c++}
template <typename T>
void f(T *a);
\end{minted}
			
			Вывод типов будет работать так - он берет тип expr без последнего указателя и подставляет его в T. В данном примере показаны простые случаи:
\begin{minted}[linenos, frame=lines, framesep=2mm, tabsize = 4, breaklines]{c++}
int x;
const int *px;

f(&x) // T - int,       ParamType - int *
f(px) // T - const int, ParamType - const int *
\end{minted}
			
			Теперь давайте рассмотрим более общий случай с указателями.
\begin{minted}[linenos, frame=lines, framesep=2mm, tabsize = 4, breaklines]{c++}
template <typename T, typename U>
void f(map<T, U*> *m);

map<int, float*>* x;
map<vector<int>, char*>* y;

f(x) // T - int,         U - float, ParamType - map<int, float*>*
f(y) // T - vector<int>, U - char,  ParamType -  map<vector<int>, char*>*
\end{minted}
		\subsubsection{Ссылки и значения}
			Теперь поговорим про передачу по ссылке и значению. То есть такие примеры
\begin{minted}[linenos, frame=lines, framesep=2mm, tabsize = 4, breaklines]{c++}
template <typename T>
void f(T a);

template <typename T>
void g(T &a);
\end{minted}
			
			Здесь используются такие правила
			\begin{enumerate}
				\item Убрать ссылочную часть исходного типа
				\item Выполняется сопоставление исходного типа с типом аргумента для определения T. (при выводе игнорируются \mintinline{c++}{const} и \mintinline{c++}{volatile})
			\end{enumerate}
			Показательным примером будет такой код
\begin{minted}[linenos, frame=lines, framesep=2mm, tabsize = 4, breaklines]{c++}
template <typename T>
void f(T x) {
	x = 10;
}

template <typename T>
void g(T &x) {
	x = 10;
}

void main() {
	int a = 5;
	int &ra = a;
	f(ra); // T - int(убирается ссылка), следовательно после f(ra) a = 5
	g(ra); // T - int, но уже у типа аргумента есть &, следовательно тип аргумента - int& и после g(ra) a = 10
}
\end{minted}
		
		\subsubsection{Function type normalizing} что это?
		
		\subsubsection{Универсальная ссылка}
			Сразу возникает вопрос что такое универсальная ссылка. Универсальная ссылка - ссылка, которая обьявляется как \textit{rvalue} ссылка, но ведет себя иначе при передаче аргументов, являющихся \textit{lvalue}.
			Для темплейтов это пишется так:
\begin{minted}[linenos, frame=lines, framesep=2mm, tabsize = 4, breaklines]{c++}
template <typename T>
void f(T&& param)
\end{minted}
			
			Для их вывода применяются такие правила:
			\begin{enumerate}
				\item Если expr представляет собой lvalue, то и T и тип аргумента будут lvalue ссылками. Интересно, что это единственный случай, когда T выводится, как ссылка.
				\item Если expr является rvalue, то у типа изначального игнорируется ссылочная часть и подставляется в T оставшееся
			\end{enumerate}
\begin{minted}[linenos, frame=lines, framesep=2mm, tabsize = 4, breaklines]{c++}
int x;
const int cx;
const int &rx;

// Здесь x, cx, rx являются lvalue
f(x)  // T - int&,       ParamType - int&
f(cx) // T - const int&, ParamType - const int&
f(rx) // T - const int&, ParamType - const int&

// 27 - rvalue
f(27) // T - int, ParamType - int&&
\end{minted}
			
			Почему происходит так, а не иначе?
			По идее должно быть правило, которое сопоставляет как взаимодействуют ссылки при T и ParamType. Всего получается четыре случая(\& и \&, \& и \&\& и так далее) В С++ есть такой механизм:
			\begin{itemize}
				\item Если где-то(или в T или в ParamType) есть \& - на выходе будет lvalue. 
				\item Если у нас два \&, то на выходе будет rvalue. 
			\end{itemize}
			Но не надо забывать, что тип может быть rvalue-ссылкой, само значение все равно будет lvalue, потому что оно имеет имя. Именно из-за этого возникает проблема прямой передачи значений(perfect forwarding), о которой будет речь далее.
			
		\subsubsection{Особенности вывода типов для шаблонов}
			\begin{enumerate}
				\item \textbf{Deduction может фэйлиться}.
				
				Давайте рассмотрим такой пример
\begin{minted}[linenos, frame=lines, framesep=2mm, tabsize = 4, breaklines]{c++}
template <typename T>
void f(T, T*);

int main()
{
	f(0, NULL);
}
				\end{minted}
				С одной стороны можно подумать что 0 - int, а NULL - нулевой указатель. Но NULL - 0, то есть int, поэтому этот код не скомпилируется с ошибкой "void f(T,T *): не удалось вывести аргумент шаблон для "T *" из "int""
				
				\item \textbf{Non-deduced context}
				
				Non-deduced context рассматривается в коде ниже и обычно нужен для того, чтобы вывести специальный тип.
\begin{minted}[linenos, frame=lines, framesep=2mm, tabsize = 4, breaklines]{c++}
template<typename T>
struct not_deduced {
	typedef T type;
};

/* Например мы хотим реализовать С-style добавление в вектор */
template<typename T>
void bad(std::vector<T> x, T value = 1);
template<typename T>
void good(std::vector<T> x, typename not_deduced<T>::type value = 1);

std::vector<std::complex<double>> x;
bad(x, 1.2);  // рассмотрим первый аргумент -
			  // входной тип std::vector<std::complex<double>>
			  // тип аргумента - std::vector<T>, следовательно T = std::complex<double>

			  // для второго входной - double, тип аргумента T, следовательно T = double
			  // мы зафэйлились :(

good(x, 1.2); // для первого тоже самое
			  // у второго есть зависимое поле следовательно он ничего не выводит.
			  // поэтому T = std::complex<double> из первого вывода
\end{minted}
				
				В данном примере мы хотели, чтобы передаваемый элемент был того же типа, что и контейнер. И non-deduced контекст помог нам в этом.
				\item \textbf{Что тут нужно?}
			
\begin{minted}[linenos, frame=lines, framesep=2mm, tabsize = 4, breaklines]{c++}

template <typename T>
void f(int);

int const a;
f(a);
\end{minted}
				
				\item \textbf{Аргументы-массивы}
				
				Часто люди думают, что массив и указатель на начало это синонимы. Это правда во множестве случаев, так как массив преобразуется в указатель на его первый элемент. Это преобразование позволяет компилироваться такому коду:
				
\begin{minted}[linenos, frame=lines, framesep=2mm, tabsize = 4, breaklines]{c++}
const char name[] = "Briggs"
const char *ptr_to_name = name;
\end{minted}
				
				Но что будет, если передать массив шаблону, принимающему параметр по значению?
				Пусть мы передали \mintinline{c++}{f(name)}
				
				Какие варианты выводов могут быть для этого?
				
\begin{minted}[linenos, frame=lines, framesep=2mm, tabsize = 4, breaklines]{c++}
void f(const char param[]); // это корректно
void f(const char *param);
\end{minted}
				
				Но так как массив рассматривается как указатель, для f(name) T будет \mintinline{c++}{const char *}.
				Но давайте рассмотрим ParamType - \mintinline{c++}{T&}. И в таком случае T будет \mintinline{c++}{const char[]} и ParamType -  \mintinline{c++}{const char (&)[]}
				
				Одним из забавных применений данной техники является вывод количества элементов, содержащихся в массиве:
				
\begin{minted}[linenos, frame=lines, framesep=2mm, tabsize = 4, breaklines]{c++}
template <typename T, std::size_t N>
constexpr std::size_t arraySize(T (&)[N]) noexcept {
	return N;
}
\end{minted}
				Модификатор constexpr делает результат функции доступным во время компиляции и можно сделать например массивы для каждого ключа свое значение.
\begin{minted}[linenos, frame=lines, framesep=2mm, tabsize = 4, breaklines]{c++}
int keyVals[] = {1, 2, 3, 4, 5, 6, 7};
std::array<int, arraySize(keyVals)> mappedVals;
\end{minted}
				
				\item \textbf{Аргументы-функции}
				
				Вместо кучи слов легче написать пример:
\begin{minted}[linenos, frame=lines, framesep=2mm, tabsize = 4, breaklines]{c++}
void someFunc(int);

template<typename T>
void f1(T param);

template<typename T>
void f2(T& param);

f1(someFunc) // тип - void(*)(int)
f2(someFunc) // тип - void(&)(int)
\end{minted}
\end{enumerate}
	
	\subsection{Вывод типов для auto}
	Если вы прочли вывод типов для шаблонов, вы знаете почти все, что следует знать о выводе типа auto, поскольку за одним любопытным исключеним типа auto представляется собой вывод типа шаблона. Как так? Существует прямая взаимосвязь между выводом типа щаблона и выводом типа auto.
	
	В шаблонах при вызове f компиляторы ипользуют expr для вывода типов T и ParamType. Когда переменная обьявлена с использованием ключевого слова auto, она играет роль T в шаблоне, а спецификатор типа переменной действует как ParamType.
	Давайте рассмотрим примеры:
\begin{minted}[linenos, frame=lines, framesep=2mm, tabsize = 4, breaklines]{c++}
auto x = 27;
const auto cx = x;
const auto& rx = x;
\end{minted}

	Для вывода типов данных штук компилятор действует так, как это бы были данные темплейты:
	
\begin{minted}[linenos, frame=lines, framesep=2mm, tabsize = 4, breaklines]{c++}
template <typename T>
void func_for_x(T param);
func_for_x(27);

template <typename T>
void func_for_cx(const T param);
func_for_cx(x);

template <typename T>
void func_for_rx(const T& param);
func_for_cx(x);
\end{minted}

	Собственно для всего того, что обьявлено выше для templates все работает также для auto.
	
	Но чем же они отличаются? До C++11 для инициализации обьектов было всего 2 возможности:
\begin{minted}[linenos, frame=lines, framesep=2mm, tabsize = 4, breaklines]{c++}
int x = 27;
int x(27);
\end{minted}
	В С++11 кроме старых вариантов появились новые варианты.
	
\begin{minted}[linenos, frame=lines, framesep=2mm, tabsize = 4, breaklines]{c++}
int x3 = {27};
int x4{27};
\end{minted}
	
	Заменив слово int на auto первые две конструкции будут вести себя также, но вторые две будут вести себя иначе. Их типом будет \mintinline{c++}{std::initializer_list<int>}. Поэтому такой код \mintinline{c++}{auto x5 = {1, 2, 3.0}} не будет компилироваться, так как не удается вывести тип T для \mintinline{c++}{std::initializer_list<T>}. Так к чему же все это? Рассмотрим такой код:
	
\begin{minted}[linenos, frame=lines, framesep=2mm, tabsize = 4, breaklines]{c++}
auto x = {1, 2, 3}; // тут все норм. Тип - std::initializer_list<int>

template <typename T>
void f(T param);

f({1, 2, 3}); // ошибка вывода типа для T
\end{minted}
	
	Однако если сделать ParamType - \mintinline{c++}{std::initializer_list<T>}, то все будет работать.
	
	Таким образом, единственное реальное различие между выводом типа для \mintinline{c++}{auto} и выводом типа шаблона заключается в том, что \mintinline{c++}{auto} предполагает, что инициализатор в фигурных скобках представляет собой \mintinline{c++}{std::initializer_list}, в то врем я как вывод типа шаблона этого не делает.
	
	Для \textit{С++11} это все, но в \textit{С++14} auto немного распростронили.\textit{ С++14 }допускает применение \mintinline{c++}{auto} для указания того, что возвращаемый тип функции должен быть выведен, а кроме того, лямбда-выражения \textit{С++14} могут использовать \mintinline{c++}{auto} в обьявлениях параметров. Однако такие штуки используют вывод типа для шаблонов. То есть такой код работать не будет:
\begin{minted}[linenos, frame=lines, framesep=2mm, tabsize = 4, breaklines]{c++}
auto createInitList() {
return {1, 2, 3}
}
\end{minted}
	
	\subsection{Использование auto}
	Часто люди думают, что использование auto - избегание неинициализированных переменных, длинных обьявлений переменных, но этим auto не ограничивается.
	\begin{enumerate}
		\item \textbf{Часто лучше определять auto, чем явно обьявлять тип.}
	
		Давайте рассмотрим такую ситуацию:
\begin{minted}[linenos, frame=lines, framesep=2mm, tabsize = 4, breaklines]{c++}
std::vector<int> v;
...
unsigned sz = v.size();
\end{minted}
		Но \mintinline{c++}{v.size()} возвращает не \mintinline{c++}{unsigned}, а \mintinline{c++}{std::vector<int>::size_type}. В 32-битной системе это 32битный тип, а в 64-битной - 64. Но \mintinline{c++}{unsigned} - всегда 32.
		Отлично, это еще только начало
		
		Давайте рассмотрим следующий код:
\begin{minted}[linenos, frame=lines, framesep=2mm, tabsize = 4, breaklines]{c++}
std::unordored_map<std::string, int> m;
...
for (const std::pair<std::string, int>& p : m) {
...
}
\end{minted}
		Выглядит нормально, но на самом деле тут происходит копирование: в мапе тип хранения \mintinline{c++}{std::pair<const std::string, int>}, поэтому он ищет преобразование \mintinline{c++}{std::pair<const std::string, int>} в \mintinline{c++}{std::pair<std::string, int>} и создается временный обьект.
		
	    \item \textbf{Не надо использовать auto всегда}
	    
		Сделаем такой код. Пусть у нас есть функция, которая возвращает \mintinline{c++}{std::vector<bool>} и нас интересует 5 бит, который мы передадим дальше
\begin{minted}[linenos, frame=lines, framesep=2mm, tabsize = 4, breaklines]{c++}
bool res = f()[4];
g(res);
\end{minted}
		Тут все нормально. А теперь давайте поменяем \mintinline{c++}{bool} на \mintinline{c++}{auto}, и теперь у нас все не будет работать:)
		
		\mintinline{c++}{operator[]} у \mintinline{c++}{vector<bool>} возвращает не \mintinline{c++}{bool&} а специальную штуку \mintinline{c++}{std::vector<bool>::reference}, которая указывает на конкретный бит. Она хранит по факут указатель и смещение. Но у нас \mintinline{c++}{vector} оказывается временным и при вызове g он будет уничтожен, собственно res будет висячим:)
		\item Использовать \mintinline{c++}{for (auto &&e : v)} - зачем это нужно?
	\end{enumerate}

	\subsection {Decltype}
		Что такое \mintinline{c++}{decltype}? Это конструкция, которая для данного имени или выражения сообщает тип этого имени или выражения
		
		Начнем с типичных случаев, в которых нет никаких подводных камней. В отличие от \mintinline{c++}{auto} и \mintinline{c++}{template} - \mintinline{c++}{decltype} возвращает точный тип имени или выражения, которое вы передаете ему:
		
\begin{minted}[linenos, frame=lines, framesep=2mm, tabsize = 4, breaklines]{c++}
const int i;             // decltype(i)    = const int
bool f(const Widget& w); // decltype(f)    = bool(const Widget&)
vector<int> v;           // decltype(v)    = vector<int>
if (v[0] == 0) ...       // decltype(v[0]) = int&
\end{minted}
		
		Основное применение \mintinline{c++}{decltype} в \textit{С++11} - обьявление шаблонов функций, в которых возвращаемый тип функции зависит от типов ее параметров.
\begin{minted}[linenos, frame=lines, framesep=2mm, tabsize = 4, breaklines]{c++}		
template<typename... As>
ReturnType f(As&& ... a) {
	return g(forward<As>(a)...);
}
\end{minted}
		То есть нам надо вернуть что-то, что возвращает функция \textbf{g}. Логично бы было написать вместо ReturnType \mintinline{c++}{decltype(g(forward<As>(a)...))}, но это не является правильным, потому что условно когда компилятор считывает \mintinline{c++}{decltype} он еще ничего не знает что такое \textbf{а}.
		
		Поэтому в \textit{С++11} был придуман новый синтаксис - завершающий возвращаемый тип. Он уже позволяет указывать в спецификации возвращаемого типа параметры функции. Вместо того, что было написано наверху можно написать так.
		
\begin{minted}[linenos, frame=lines, framesep=2mm, tabsize = 4, breaklines]{c++}
template<typename... As>
auto f(As&& ... a) -> decltype(g(forward<As>(a)...)) {
	return g(forward<As>(a)...);
}
\end{minted}
		
		Многие могли подумать, что вместe c \mintinline{c++}{auto} тут подразумевается вывод типов, но в \textit{С++11} это не так. Оно просто означает, что возвращаемый тип будет написан после аргументов. В C++14 мы можем опустить заверщающий возвращаемый тип, оставляя только одно ведущее ключевое слово \mintinline{c++}{auto}. При таком обьявлении \mintinline{c++}{auto} означает, что имеет место \textit{вывод типа}, а точнее \textit{вывод типа шаблона}
		
		Это может сыграть плохую шутку в том случае, когда например возвращаемый тип представляет собой ссылку.
\begin{minted}[linenos, frame=lines, framesep=2mm, tabsize = 4, breaklines]{c++}
template<typename Container, typename index>
auto authAndAccess(Container& c, Index i) {
	authenticateUser();
	return c[i];
}
\end{minted}
		Получается, что он проигнорирует ссылку и такой код:
\begin{minted}[linenos, frame=lines, framesep=2mm, tabsize = 4, breaklines]{c++}
std::deque<int> d;
...
authAndAccess(d, 5) = 10;
\end{minted}
		Не будет компилироваться.
		
		Чтобы заставить \mintinline{c++}{authAndAccess} работать так, как мы хотим(чтобы можно было писать код сверху), нам надо использовать для ее возвращаемого типа выдов типа \mintinline{c++}{decltype}, т.е. указать, что \mintinline{c++}{authAndAccess} должна возвращать в точности тот же тип, что и выражение \mintinline{c++}{c[i]}. Добрые люди, предвидя необходимость использования в некоторых случаях правила вывода типа decltype, сделали это возможным в \textit{С++14} с помощью спецификатора \mintinline{c++}{decltype(auto)}.
		
		В этом есть смысл: \mintinline{c++}{auto} указывает, что тип должен быть выведен, а \mintinline{c++}{decltype} говорит о том, что в процессе вывода следует использовать правила \mintinline{c++}{decltype}. \\\\
		Использование \mintinline{c++}{decltype(auto)} не ограничивается возвращаемым типом функции. Он может быть удобен для обьявления переменны, когда хотим применять правила вывода типа decltype к инициализирующему выражению:
\begin{minted}[linenos, frame=lines, framesep=2mm, tabsize = 4, breaklines]{c++}
Widget w;
const Widget& cw = w;
auto myWidget1 = cw;           // Widget
decltype(auto) myWidget2 = cw; // const Widget&
\end{minted}
		Теперь давайте перейдем к подводным камням \mintinline{c++}{delctype}. В начале параграфа говорилось, что \mintinline{c++}{decltype} для данного имени или выражения сообщает тип этого \textbf{имени или выражения}. То есть есть разница между a и (a):
\begin{minted}[linenos, frame=lines, framesep=2mm, tabsize = 4, breaklines]{c++}
int a;
decltype(a)   // int
decltype((a)) // int&
\end{minted}
		Во втором случае у нас получается не имя, а выражение за счет скобочек, а так как внутри скобочек находится \textit{lvalue}, поэтому тип возращаемый будет \textit{lvalue}.
		
		И это может чувствоваться на таком примере.
\begin{minted}[linenos, frame=lines, framesep=2mm, tabsize = 4, breaklines]{c++}
		decltype(auto) f1() {
			int x = 0;
			...
			return x;
		}
		
		decltype(auto) f2() {
			int x = 0;
			...
			return (x);
		}
\end{minted}
		Вроде бы разница между функциями небольшая, но f2 является бомбой замедленного действия, поведение которого нельзя предсказать. Поэтому с \mintinline{c++}{decltype} надо быть предельно внимательным, потому что малейшая неточность может привести к плохим последствиям.
